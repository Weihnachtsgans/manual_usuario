\documentclass{article}
\usepackage[utf8]{inputenc}

\title{Manual de usuario Turismo App}
\date{Abril 2018}
\author{Aguilera, Mayerlin C.I: 24.290.965,\\
        Castañeda, Jhosevic C.I: 26.960.725, \\
        Mendoza, Víctor C.I: 21.476.548, \\
        Mota, Sergio C.I: 21.098.290}

\usepackage{natbib}
\usepackage{graphicx}
\usepackage{imakeidx}
\makeindex

\begin{document}

\maketitle
\newpage
\tableofcontents
\newpage

\section{Página principal}
Al abrir el programa, se mostrará una interfaz por consola con las posibles opciones a ejecutar. Pudiendo volver atrás siempre que se quiera, ingresando "GO BACK!!!" en alguno de los campos
//Inserte figura del menú principal
\subsection{Login}
A la hora de logearse, se pedirá ingresar las credenciales del usuario para comprobar si sus datos son correctos, y, dependiendo se sus privilegios, se pasará de un menú a otro.

\subsection{Register}
Se le preguntará al usuario el nombre de usuario y la contraseña deseada. Asimismo se le asignarán valores genéricos para el resto de los campos, los cuales deberán ser cambiados posteriormente antes de realizar cualquier transacción con la aplicación.

\section{Menú usuario}
Al ingresar a la aplicación, se le permitirá al usuario cambiar sus datos por unos más acordes a su gusto. Lo único que no se permitirá cambiar, será el id de usuario.

\section{Menú de administrador}
Al administrador, se le permiten las siguientes acciones:
\begin{enumerate}
    \item Registrar ruta
    \item Registrar hospedaje
    \item Reportes
\end{enumerate}

\subsection{Registrar ruta}
El administrador puede registrar una ruta nueva, siempre y cuando suministre los datos necesarios, separados por un punto y coma (;) cada uno.
En caso de haber ocurrido un error durante el registro, se le notificará al usuario.

\subsection{Registrar hospedaje}
El administrador puede registrar un nuevo hospedaje, siempre y cuando suministre los datos necesarios para dicha acción, los cuales deben estar separados con un punto y coma (;).

\subsection{Reportes}
Se pueden realizar las siguientes acciones:
\begin{enumerate}
    \item Total de visitas por destino
    \item Proveedores de transporte por empresa
    \item Proveedores de residencia por empresa
\end{enumerate}
\paragraph{Total de visitas por destino}\mbox{}\\
Se muestra por pantalla un conjunto de tuplas, que indica el destino turístico en cuestión y la cantidad de visitas para dicho destino.

\paragraph{Proveedores de transporte por empresa}\mbox{}\\
Se muestra todos los proveedores de transporte que han contratado una determinada empresa, ordenados ascendentemente por la cantidad de veces que se ha contratado.

\paragraph{Proveedores de residencia por empresa}\mbox{}\\
Se muestra todos los proveedores de residencia que han contratado una determinada empresa, ordenados ascendentemente por la cantidad de veces que se ha contratado.

\end{document}

